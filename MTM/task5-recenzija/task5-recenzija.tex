\documentclass{article}

\usepackage[L7x,T1]{fontenc}
\usepackage[utf8]{inputenc}
\usepackage{csquotes}
\usepackage[english]{babel}
\usepackage[style=authoryear]{biblatex}
\addbibresource{bib.bib}
\usepackage{hyperref}

\title{
    Research Methodology -- Fifth exercise\\ \vspace{4mm}
    Critical evaluation of scientific work
}

\author{Motiejus Jakštys}

\date{\today}

\begin{document}
\maketitle

\section{Introduction}

This article reviews \cite{186171}, and answers the following questions:

\begin{enumerate}
    \item What kind of study is this? Theoretical, strategic, applied, or
        experimental?
    \item What is the main purpose of the research task (descriptive,
        explanatory, correlative, prognostic, prescriptive, or exploratory?
    \item What strategies have been applied? Qualitative, quantitative or
        mixed?
    \item Do the findings adequately reflect the results?
    \item Has the scientific method been applied properly?
    \item Are the findings based on the research findings described in the
        text?
    \item Can the study be repeated, is there sufficient information?
    \item Did the study create new knowledge? Is there practical value?
\end{enumerate}

\section{The Paper}

Besides other things, \cite{186171} analyzed 198 randomly selected failures in
popular distributed systems, and classified the reasons for each failure. This
is one of the most interesting findings:

\blockquote[\cite{186171}] {
    Almost all (92\%) of the catastrophi system failures are the result of
    incorrect handling of non-fatal errors explicitly signaled in software.
}

\section{Structure overview}

\subsection{Kind of study}
The paper is strategic, applied:

\begin{description}

    \item[Strategic:] authors have developed an artifact \tt{Aspirator} which
        helps software maintainers find certain classes of bugs. What is more,
        they provided new knowledge, like in the quote above.

    \item[Applied:] the artifact of the work, \tt{Aspirator}, can be applied by
        other software developers looking for similar classes of bugs.

\end{description}

\subsection{Purpose of the research task}

The research task is descriptive and correlative: given a well-understood
situation of distributed systems fail catastrophically, researchers are finding
common reasons for failures, and developing tools to mitigate them.

Conclusions and suggestions are prescriptive: the researchers are warning
engineers against common failures, and suggesting tools to mitigate them.

\subsection{Applied Strategies}

The task is mixed:
\begin{description}
    \item[Quantitative:] researchers are analyzing and classifying a large
        number of bugs.
    \item[Qualitative:] each bug requires careful analysis in order to classify
        it and make interesting conclusions.
\end{description}

\subsection{Do findings reflect the purpose and results?}

Research findings are derived directly from the purpose and research results.
Namely, the researchers set out to find the most common reasons for
catastrophic failures in distributed systems. They found them, classified them,
and gave suggestions for future generations of distributed systems developers.

\subsection{Scientific Method}

\printbibliography

\end{document}
