\documentclass{article}

\usepackage[L7x,T1]{fontenc}
\usepackage[utf8]{inputenc}
\usepackage[lithuanian]{babel}
\usepackage{hyperref}

\usepackage[style=authoryear]{biblatex}
\addbibresource{bib.bib}


\title{
    Mokslinių tyrimų metodologija\\
    Penktoji užduotis\\ \vspace{4mm}
    Mokslo Darbo Kritinis Vertinimas
}

\author{Motiejus Jakštys}

\date{\today}

\begin{document}
\maketitle


\section{Klausimai}
    Kokio tai tipo tyrimas (teorinis, strateginis,  taikomas, eksperimentinis)?
    Kokio tipo pagrindinis tyrimo uždavinys (aprašomasis, aiškinamasis, koreliacinis, prognostinis, nurodomasis, žvalgomasis).
    Kokios taikytos tyrimo strategijos (kiekybinės, kokybinės, mišrios)?

    Ar pasiekti rezultatai atitinka darbo tikslą ir uždavinius?
    Ar išvados tinkamai atspindi rezultatus?
    Ar tinkamai taikytas mokslinis metodas?
    Ar išvados yra pagrįstos tekste aprašytais tyrimo rezultatais?
    Ar tyrimas gali būti pakartotas, ar tam pakanka informacijos?
    Ar tyrimas sukūrė naujas žinias? Ar yra pademonstruota  praktinė vertė?


According to \cite{186171}, testing error conditions is critical for robust
distributed systems.

\printbibliography

\end{document}
