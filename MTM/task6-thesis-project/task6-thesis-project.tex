\documentclass{article}

\usepackage[L7x,T1]{fontenc}
\usepackage[utf8]{inputenc}
\usepackage{csquotes}
\usepackage[lithuanian]{babel}
\usepackage[style=authoryear]{biblatex}
\addbibresource{bib.bib}
\usepackage{hyperref}
\usepackage{caption}
\usepackage{subcaption}

\title{
    Prieinamų orų prognozių palyginimas Lietuvoje ir Baltijos šalyse \\ \vspace{4mm}

    \large Mokslinių tyrimų metodologija\\
    Šeštoji užduotis -- Magistrinio darbo projektas
}

\author{Motiejus Jakštys}

\date{\today}

\begin{document}
\maketitle

\section{Santrauka}

Norėdami sužinoti oro prognozę, nežinome, kuris iš daugelio siūlomų šaltinių
yra tiksliausias. Pasitikėdami netiksliu šaltiniu rizikuojame padaryti
neefektyvius sprendimus: žemės ūkyje, infrastruktūros projektuose, arba
kasdieniame gyvenime. Šis darbas pasakys, kuri orų prognozė Lietuvoje tirtuoju
periodu buvo tiksliausia. Paviešinę šiuos rezultatus galbūt atkreipsime tiekėjų
dėmesį į jų prognozių tikslumą, ir padėsime skaičiuosiems žmonėms išsirinkti
geriausią orų prognozės tiekėją.

\section{Įžanga}

Norėdami sužinoti oro prognozę, galime pasirinkti keletą šaltinių: žinios per
televiziją ar radiją, nacionalinė hidrometeorologijos tarnybos interneto
svetainė, arba vieną iš aibės pasaulinių interneto svetainių. Žmonės renkasi
pagal patogumą ir įpročius, tačiau nebūtinai jų pasirinktas šaltinis yra
objektyviai geriausias.

Kadangi orų prognozių tikslumas nėra lengvai prieinamas ir ne daug tiriamas,
įprasta rinktis pagal tai, kas žinoma ir matoma -- vartotojo sąsajos patogumas
arba anekdotiniai potyriai ("šis šaltinis dar niekada manęs neapvylė, o
tavęs?").

Šis darbas per kalendorinius metus surinks populiarių Lietuvoje šaltinių 1, 5
ir 10 dienų orų prognozes, palygins jas su faktiniais orais, ir atsakys į šiuos
klausimus:

\begin{itemize}
    \item Ar orų prognozių tikslumas skiriasi tarp regionų?
    \item Kuris tiekėjas tiksliausias tam tikrame regione?
    \item Kuris tiekėjas tiksliausias tam tikru metų laiku?
    \item Kuris tiekėjas tiksliausiai prognozuoja kritulius, minimalią
        temperatūrą, maksimalią temperatūrą, vėjo greitį?
\end{itemize}

Jei pavyks surinkti duomenis iš kitų valstybių nacionalinių tiekėjų:
\begin{itemize}
    \item Kiek skiriasi to paties tiekėjo orų prognozės tarp valstybių (pvz.,
        meteo.pl duoda orų prognozes ir Vilniui)?
    \item Kiek skiriasi "nacionalinių" tiekėjų orų prognozių patikimumas
        prognozuojant orus jų pačių valstybėje? Pvz., Lietuvos, Lenkijos ir
        Vokietijos?
\end{itemize}

\section{Ankstesni tyrimai}

Orų prognozių tikslumas 11 metų laikotarpyje buvo analizuotas
\cite{rose2017analysis}, tačiau buvo neskirstoma tarp regionų: išvadose buvo
visi tirti kontinentai. Šiame darbe norime skirstyti prognozių tikslumą pagal
regionus, taip pat pridėti "nacionalinius" tiekėjus.

\cite{rose2017analysis} taip pat aprašytas duomenų rinkimo ir interpretavimo
būdas. Kad galėtume palyginti savo tyrimo skaičius su tuo darbu, duomenų
rinkimo ir interpretavimo metodiką naudosime tą pačią.

\printbibliography

\end{document}
