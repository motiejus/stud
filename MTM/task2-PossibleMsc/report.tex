\documentclass{article}

\usepackage[L7x,T1]{fontenc}
\usepackage[utf8]{inputenc}
\usepackage[lithuanian]{babel}
\usepackage{hyperref}

\title{Mokslinių tyrimų metodologija\\ \vspace{4mm} 
Antroji užduotis -- magistro darbų problemos}

\author{\bf Motiejus Jak\v{s}tys\\ \bf}

\date{\today}

\begin{document}
\maketitle

\section{TYRIMO PROBLEMA}

Elektros laidai, nutiesti miškuose, turi būti reguliariai tikrinami, kad jų
nepasiektų medžių šakos ar viršūnės. Įprastinėmis priemonėmis tai yra
sudėtingas darbas dėl riboto medžių aukščio matomumo. Šiame darbe įvertinsime,
ar naudojant bepiločius orlaivius procesą galima paefektyvinti.


\section{TYRIMO TIKSLAS}

Rodiklis susideda iš dviejų dalių:
\begin{itemize}

    \item Kaina ištirti dabartinę situaciją. Matuojama žmogaus valandomis
        kilometrui.

    \item Gavus papildomus duomenis iš bepiločių orlaivių numatyti, kiek laiko
        atkarpa dar gali būti netrikrinama, taip sumažinant ateities kaštus.

\end{itemize}


\section{TYRIMO KLAUSIMAS}

Ar naudojant bepiločius orlaivius galima efektyviau nustatyti laidų miške
blokavimą?

\section{TYRIMO TEMA}


Ištirti medžių augimo riziką elektros laidams naudojant bepiločius orlaivius.

\section{NUORODOS}

\begin{itemize}
    \item Changes affecting generalization of land cover features in a smaller scale.
    \item Applicability of Unmanned Aerial Vehicles in Research on Aeolian Processes.
    \item Automatizuotas hidrografijos kanalu išskyrimas lietuvos georeferencinio pagrindo duomenu bazeje.
\end{itemize}

\end{document}
