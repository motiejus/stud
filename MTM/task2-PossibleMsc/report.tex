\documentclass{article}

\usepackage[L7x,T1]{fontenc}
\usepackage[utf8]{inputenc}
\usepackage[lithuanian]{babel}

\title{Mokslinių tyrimų metodologija\\ \vspace{4mm} 
Antroji užduotis -- magistro darbų problemos}

\author{\bf Motiejus Jak\v{s}tys\\ \bf}

\date{\today}

\begin{document}
\maketitle

\section{TYRIMO PROBLEMA}

Konkrečiai suformuluota mokslinio tyrimo reikalaujanti ir neišspręsta problema.
Pagrindimas, kad jai spręsti reikia mokslinių tyrimų. Bendras situacijos
įvertinimas, paremtas literatūros apžvalga. 

\section{TYRIMO TIKSLAS}

Kokio teigiamo pokyčio galėtumėt pasiekti savo tyrimu. Įvardykite norimą pokyti
ir jo objektyvų rodiklį (rodiklius)

\section{TYRIMO KLAUSIMAS}

Klausimas, į kurį tikitės atsakyti, atlikę tyrimą. Operacionalizuokite sąvokas
- tiksliai jas apibrėžkite, nurodykite kaip bus matuojami parametrai.  

\section{TYRIMO TEMA}

Temos pasiūlymas, kurį pateiksite pasirinktam vadovui. 

\end{document}
