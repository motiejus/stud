\documentclass{article}

\usepackage[L7x,T1]{fontenc}
\usepackage[utf8]{inputenc}
\usepackage{csquotes}
\usepackage[lithuanian]{babel}
\usepackage[maxbibnames=99,style=authoryear]{biblatex}
\addbibresource{bib.bib}
\usepackage{hyperref}
\usepackage{caption}
\usepackage{subcaption}

\title{
    Prieinamų orų prognozių palyginimas Lietuvoje ir Baltijos šalyse \\ \vspace{4mm}

    \large Mokslinių tyrimų metodologija\\
    Septintoji užduotis -- Magisto darbo tyrimų planas
}

\author{Motiejus Jakštys}

\date{\today}

\begin{document}
\maketitle

\newpage

\section{Planuojamo tyrimo aktualumas, originalumas, teorinė ir praktinė reikšmė}

\subsection{Aktualumas}

Kaip aprašyta \cite{motiejus-task6}, darbas aktualus žmonėms ir industrijoms,
kurie/kurios vykdo veiklą lauke šio darbo tiriamoje teritorijoje.

\subsection{Originalumas}

\cite{rose2017analysis} ištyrė orų prognozių tiekėjus globaliu mastu. Tačiau tas darbas
nenurodo prognozių tikslumo vietiniu mąstu; taip pat duomenys nėra viešai prieinami.

Šis darbas bus aktualus norint įvertinti prognozės tikslumą konkrečiame regione.

\section{Tyrimo uždaviniai}
{\small Pagrindinės veiklos, kurias reikia atlikti siekiant tikslo}

Darbas susideda iš kelių dalių:
\begin{description}
    \item[Duomenų surinkimas] iš duomenų portalų. Kadangi norime įvertinti
        kalendorinių metų ciklą, duomenis reikės rinkti iš kelių tiekėjų visus
        metus, bent po vieną kartą per dieną.
    \item[Duomenų, apdorojimas]: portaluose prognozės pateikiamos skirtingu
        formatu. Kad galėtume prognozes lyginti, duomenis reikia suvienodinti.
    \item[Analizės etapas] prasidės, kai kalendorinių metų duomenys bus surinkti
        ir susisteminti. Kaip paminėta \cite{motiejus-task6}, naudosime "naivų"
        metodą palyginimui su \cite{rose2017analysis}, ir erdvinius metodus iš
        \cite{verification2015}.
    \item[Atvaizdavimas]: kokiu būdu pateikti žemėlapį, geriausiai perteikiantį
        analizės rezultatus?
    \item[Išvados]: kuri prognozės sistema patikimiausia? Kiek patikimumas
        priklauso nuo regiono?
\end{description}

\section{Laukiami rezultatai}
{\small Naujos žinios, prognozė, rekomendacijos ar kt.}
3. Apibrėžkite laukiamus rezultatus (naujos žinios, prognozė, rekomendacijos ar kt.).

\section{Kokio tipo tai bus tyrimas, kokie reikalingi ištekliai}
4. Įvertinkite kokio tipo tai bus tyrimas, kokie reikalingi ištekliai, kokios yra Jūsų stiprybės ir silpnybės, į kurias reikia atsižvelgti jį vykdant.

\section{Stiprybės ir silpnybės, į kurias reikia atsižvelgti jį vykdant}
4. Įvertinkite kokio tipo tai bus tyrimas, kokie reikalingi ištekliai, kokios yra Jūsų stiprybės ir silpnybės, į kurias reikia atsižvelgti jį vykdant.

\section{Naudojami pradiniai duomenys ir pagrindinės numatomų taikyti metodų grupės}
5. Aprašykite naudojamus pradinius duomenis ir pagrindines numatomų taikyti metodų grupes.

\printbibliography

\end{document}
