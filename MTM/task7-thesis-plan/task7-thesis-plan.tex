\documentclass{article}

\usepackage[L7x,T1]{fontenc}
\usepackage[utf8]{inputenc}
\usepackage{csquotes}
\usepackage[lithuanian]{babel}
\usepackage{hyperref}
\usepackage{caption}
\usepackage{subcaption}

\title{
    Prieinamų orų prognozių palyginimas Lietuvoje ir Baltijos šalyse \\ \vspace{4mm}

    \large Mokslinių tyrimų metodologija\\
    Septintoji užduotis -- Magisto darbo tyrimų planas
}

\author{Motiejus Jakštys}

\date{\today}

\begin{document}
\maketitle

\newpage

\section{Planuojamo tyrimo aktualumas, originalumas, teorinė ir praktinė reikšmė}
1. Įvertinkite planuojamo tyrimo aktualumą, originalumą, teorinę ir praktinę reikšmę.

\section{Tyrimo uždaviniai}
{\small Pagrindinės veiklos, kurias reikia atlikti siekiant tikslo}
2. Apibrėžkite tyrimo uždavinius (pagrindines veiklas, kurias reikia atlikti siekiant tikslo)

\section{Laukiami rezultatai}
{\small Naujos žinios, prognozė, rekomendacijos ar kt.}
3. Apibrėžkite laukiamus rezultatus (naujos žinios, prognozė, rekomendacijos ar kt.).

\section{Kokio tipo tai bus tyrimas, kokie reikalingi ištekliai}
4. Įvertinkite kokio tipo tai bus tyrimas, kokie reikalingi ištekliai, kokios yra Jūsų stiprybės ir silpnybės, į kurias reikia atsižvelgti jį vykdant.

\section{Stiprybės ir silpnybės, į kurias reikia atsižvelgti jį vykdant}
4. Įvertinkite kokio tipo tai bus tyrimas, kokie reikalingi ištekliai, kokios yra Jūsų stiprybės ir silpnybės, į kurias reikia atsižvelgti jį vykdant.

\section{Naudojami pradiniai duomenys ir pagrindinės numatomų taikyti metodų grupės}
5. Aprašykite naudojamus pradinius duomenis ir pagrindines numatomų taikyti metodų grupes.

\end{document}
