\documentclass{article}

\usepackage[L7x,T1]{fontenc}
\usepackage[utf8]{inputenc}
\usepackage{csquotes}
\usepackage[english]{babel}
\usepackage[maxbibnames=99,style=authoryear]{biblatex}
\addbibresource{bib.bib}
\usepackage{hyperref}
\usepackage{caption}
\usepackage{subcaption}
\usepackage{gensymb}
\usepackage{varwidth}
\usepackage{tikz}
\usetikzlibrary{er,positioning}

\title{
    Cartografic Generalization of Lines \\
    (example of rivers) \\ \vspace{4mm}
}

\author{Motiejus Jakštys}

\date{\today}

\begin{document}
\maketitle

\newpage

\section{Abstract}
\label{sec:abstract}

Ready-to-use, open-source line generalization solutions emit poor cartographic
output. Therefore, if one is using open-source technology to create a
large-scale map, downscaled lines (e.g. rivers) will look poorly. This paper
explores line generalization algorithms and suggests to implement an algorithm
for an avid GIS developer. Once the algorithm is implemented and integrated to
open-source GIS solutions (e.g. QGIS), rivers on future large-scale maps will
look professionally downscaled.

\section{Introduction}
\label{sec:introduction}

\section{The Problem}
\label{sec:the_problem}

\section{My Idea}
\label{sec:my_idea}

\section{The Details}
\label{sec:the_details}

\section{Related Work}
\label{sec:related_work}

\section{Conclusions and Further Work}
\label{sec:conclusions_and_further_work}

\printbibliography

\end{document}
