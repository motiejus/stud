\documentclass{article}

\usepackage[L7x,T1]{fontenc}
\usepackage[utf8]{inputenc}
\usepackage{csquotes}
\usepackage[lithuanian]{babel}
\usepackage[maxbibnames=99,style=authoryear]{biblatex}
\addbibresource{bib.bib}
\usepackage{hyperref}
\usepackage{caption}
\usepackage{subcaption}
\usepackage{gensymb}
\usepackage{varwidth}
\usepackage{tikz}
\usetikzlibrary{er,positioning}

\title{
    Žemėlapio temos bei kartografinio vaizdavimo būdų aprašymas \\ \vspace{4mm}

    \large Kartografinės komunikacijos pagrindai\\
    Pirmoji užduotis
}

\author{Motiejus Jakštys}

\date{\today}

\begin{document}
\maketitle

\newpage

\section{Pavadinimas ir tema}
\label{sec:tema}

\textbf{Žemėlapio pavadinimas: A. Vienuolio gimnazija -- Sportinis Žemėlapis.}

Žemėlapio idėja kilo orientavimosi sporto entuziastui Rolandui Jakščiui
norint supažindinti 4 klasių moksleivius su orientavimosi sportu.

Patogiausias būdas supažindinti klasę moksleivių su orientavimosi sportu --
suorganizuoti nedideles varžybas per kūno kultūros pamoką. Logistiškai tam
reikia arba vežti vaikus prie tikros orientavimosi trasos (kas yra sudėtinga su
20+ 10-11 metų moksleivių), arba padaryti varžybas jų mokyklos kieme.

Kad padarytume varžybas mokyklos kieme, reikalingas mokyklos orientavimosi
žemėlapis.

Taip ir atsirado realus edukacinis poreikis šiam žemėlapiui.

\subsection{3D vizualizacija}

Kai orientavimosi sporto žemėlapis atitiks poreikius, planuojamas ir antrasis
etapas: 3D žemėlapis iš aukščio taškų.

\section{Pagrindiniai žemėlapio parametrai}

\begin{description}
    \item[Teritorija:] Vilniaus A. Vienuolio mokyklos kiemas. Kvartalas,
        apribotas šiomis gatvėmis:
        \begin{itemize}
            \item Pylimo.
            \item Bazilijonų.
            \item Aušros Vartų.
            \item Geležinkelio.
        \end{itemize}
    \item[Mastelis:] 1:2000.
    \item[Maketo dydis:] A5.
\end{description}

\section{Duomenų šaltiniai}

\begin{description}
    \item[Topografinis pagrindas:] Didelio tikslumo Vilniaus topografinė
        duomenų bazė \texttt{TDB500V}.

    \item[Isohipsės iš:]  LR teritorijos skaitmeniniai erdviniai žemės
        paviršiaus lazerinio skenavimo taškų duomenys \texttt{SEŽP\_0.5LT}.

\end{description}

Kai kurie parametrai, svarbūs orientavimosi sportui, topografiniame žemėlapyje
nenurodyti. Pvz., ar iš vienos teritorijos galima pereiti į kitą, ar krūmų
tinklas mokyklos parkelyje yra praeinamas, ar ne (tai orientavimosi žemėlapyje
reikia žymėti skirtingais ženklais). Todėl, sudarant žemėlapį, autoriai kelis
kartus važiavo gyvai apžiūrėti vietos.

\section{Kartografinio pagrindo elementai}

\begin{itemize}
    \item Žmogaus sukurti objektai:
        \begin{itemize}
            \item Pastatai.
            \item Keliai.
            \item Šaligatviai.
            \item Stovėjimo aikštelės.
            \item Tvoros.
            \item Gyvatvorės.
        \end{itemize}
    \item Gamtos objektai:
        \begin{itemize}
            \item Reljefas.
            \item Pavieniai medžiai.
        \end{itemize}
\end{itemize}

\section{Vaizdavimo būdai}

Šio žemėlapio vaizdavimo būdas gana aiškiai apibrėžtas: naudojamas ISOM2007
standartas \cite{isom2007}.

Isohipsių sugeneravimui reikėjo apdoroti \texttt{SEŽP\_0.5LT} taškų masyvus šiais žingsniais:
\begin{itemize}
    \item Interpoliuoti: paversti pavienius taškus į rastrinį paveikslėlį.
    \item Sudaryti kontūrus.
    \item Sušvelninti kontūrus.
    \item Suapvalinti kontūrus.
\end{itemize}

Tuomet reikėjo atrinkti ir išmesti mažyčius kontūrus, kurie rodo labai mažą
lokalų pakilimą. Tokie "kalneliai" itin apkrauna žemėlapį labai mažai
suteikdami informacijos, todėl buvo pašalinti.

\section{GIS įranga}

Darbas padalintas į du etapus: orientavimosi žemėlapis ir 3D modelis.

Pirmajam etapui naudota GIS įranga:

\begin{description}
    \item[Kartografiniam pagrindui sudaryti:] QGIS.
    \item[Isohipsėms sugeneruoti] QGIS.
    \item[Orientavimosi žemėlapiui piešti:] \cite{mapper}.
\end{description}

Antrajam etapui planuojama naudoti D3.js.

\printbibliography

\end{document}
