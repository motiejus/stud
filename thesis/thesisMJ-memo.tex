\documentclass{article}

\usepackage[L7x,T1]{fontenc}
\usepackage[utf8]{inputenc}
\usepackage[lithuanian]{babel}
\usepackage{hyperref}
\usepackage{cite}

\title{Galimo magistrinio darbo aprašymas\\ \vspace{4mm} 
Oro prognozės tikslumo palyginimas pagal geografiją}

\author{Motiejus Jakštys}

\date{\today}

\begin{document}
\maketitle

\section{Santrauka}

\section{Įžanga}
Norėdami sužinoti oro prognozę, galime pasirinkti keletą šaltinių: žinios per
televiziją ar radiją, nacionalinė hidrometeorologijos tarnybos interneto
svetainė, arba viena iš aibės užsienietiškų interneto svetainių. Žmonės renkasi
pagal patogumą ir įpročius, tačiau nebūtinai jų pasirinktas šaltinis yra
tiksliausias.

Orų prognozių tikslumas retai analizuojamas, todėl žmonės renkasi pagal tai, ką
žino ir mato -- vartotojo sąsajos patogumas arba anekdotiniai potyriai iš
anksčiau ("šis šaltinis dar neapvylė").

Šis darbas per kalendorinius metus surinks populiarių šaltinių 1, 5 ir 10 dienų
orų prognozes ir palygins prognozes su faktiniais orais.

Išanalizavus duomenis, bus galima atsakyti į klausimus:
\begin{itemize}
    \item Ar orų prognozių tikslumas skiriasi per regionus?
    \item Kuris tiekėjas tiksliausias tam tikrame regione?
    \item Kuris tiekėjas tiksliausias tam tikru metų laiku?
\end{itemize}

Jei pavyks surinkti duomenis iš daugiau nacionalinių tiekėjų iš kitų šalių:
\begin{itemize}
    \item Kiek skiriasi to paties tiekėjo orų prognozės tarp šalių?
    \item Kiek skiriasi "nacionalinių" tiekėjų orų prognozių patikimumas? Pvz.,
        Lietuvos, Lenkijos ir Vokietijos.
\end{itemize}

\section{Ankstesni darbai ir tyrimo metodika}

Orų prognozių tikslumas per paskutinius 12 metų buvo detaliai
išanalizuotas\cite{rose2017analysis}, tačiau visa analizė ir išvados buvo
daromos globaliai, neskirstant tarp regionų: į tą pačią analizę buvo sudėti
visi tirti kontinentai.

Šiame darbe taip pat aprašytas duomenų rinkimo ir interpretavimo būdas. Kad
galėtume palyginti šio tyrimo skaičius su ta įmone, duomenų rinkimo ir
interpretavimo metodiką naudosime tą pačią.

\bibliography{bib}{}
\bibliographystyle{plain}

\end{document}
