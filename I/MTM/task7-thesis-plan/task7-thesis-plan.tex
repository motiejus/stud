\documentclass{article}

\usepackage[L7x,T1]{fontenc}
\usepackage[utf8]{inputenc}
\usepackage{csquotes}
\usepackage[lithuanian]{babel}
\usepackage[maxbibnames=99,style=authoryear]{biblatex}
\addbibresource{bib.bib}
\usepackage{hyperref}
\usepackage{caption}
\usepackage{subcaption}

\title{
    Prieinamų orų prognozių palyginimas Lietuvoje ir Baltijos šalyse \\ \vspace{4mm}

    \large Mokslinių tyrimų metodologija\\
    Septintoji užduotis -- Magisto darbo tyrimų planas
}

\author{Motiejus Jakštys}

\date{\today}

\begin{document}
\maketitle

\newpage

\section{Planuojamo tyrimo aktualumas, originalumas, teorinė ir praktinė reikšmė}

\subsection{Aktualumas}

Kaip aprašyta \cite{motiejus-task6}, darbas aktualus žmonėms ir industrijoms,
kurie/kurios vykdo veiklą lauke šio darbo tiriamoje teritorijoje.

\subsection{Originalumas}

\cite{rose2017analysis} ištyrė orų prognozių tiekėjus ir pateikė išvadas per
tiekėją, imdami globalius duomenis. Tačiau šis tyrimas nenurodo, kaip keičiasi
prognozių tikslumas tarp skirtingų regionų.  Deja, duomenys nėra viešai
prieinami.

Šis darbas bus aktualus norint įvertinti prognozės tikslumą konkrečiame regione.

\section{Tyrimo uždaviniai}
Darbas susideda iš kelių dalių:
\begin{description}
    \item[Duomenų surinkimas] iš duomenų portalų. Kadangi norime įvertinti
        kalendorinių metų ciklą, duomenis reikės rinkti iš kelių bent vieną
        kartą per dieną iš kelių tiekėjų visus metus.
    \item[Duomenų apdorojimas:] portaluose prognozės pateikiamos skirtingu
        formatu. Kad galėtume prognozes lyginti, duomenis reikia suvienodinti.
    \item[Analizė] prasidės, kai kalendorinių metų duomenys bus surinkti
        ir susisteminti. Kaip paminėta \cite{motiejus-task6}, naudosime "naivų"
        metodą palyginimui su \cite{rose2017analysis}, ir erdvinius metodus iš
        \cite{verification2015}.
    \item[Išvados:] kuri prognozės sistema patikimiausia? Kiek patikimumas
        priklauso nuo regiono? Kurie tiekėjai duoda patikimiausius prognozės
        aspektus?
    \item[Vizualizacija:] kokį pateikti žemėlapį, geriausiai perteikiantį
        analizės rezultatus ir rekomendacijas?
\end{description}

\section{Laukiami rezultatai}

Gauti rezultatai padės atsakyti į hipotezes:
\begin{itemize}
    \item ar orų viešai prieinamų oro prognozių patikimumas svyruoja tarp
        regionų?
    \item kuris orų prognozės tiekėjas geriausias? Bendrai? Žiemą? Vasarą?
        Lietui? Vėjui?
\end{itemize}

\section{Tyrimo tipas, reikalingi ištekliai}

Tyrimas yra {\textbf prognostinis kiekybinis}: pagal praeities orų prognozių
rezultatus ekstrapoliuojame, kurie tiekėjai kuriuose regionuose orus numato
geriausiai. Kiekybinis, nes aiškinamasi ne priežastys, o rezultatas.

Kadangi tyrimas remsis duomenimis iš reguliariai atsisiunčiamų viešai priimamų
šaltinių, tokie ir bus ištekliai. Svarbu, kad duomenų surinkimo metu
interneto svetainės, iš kurių bus siunčiami duomenys, išliktų prieinamos, ir
surenkamų duomenų formatas iš esmės nepasikeistų.

\section{Stiprybės ir silpnybės}

Duomenų kiekis bus toks, kad surinkimą ir analizę reikės automatizuoti.
{\textbf Stiprybė}: esu programuotojas ir mokėsiu tai daryti.

Silpnybė: tiriu meteorologinius reiškinius, tačiau nesu meteorologas ir neturiu
net meteorologijos pagrindų. Tačiau darbo vadovas yra ekspertas meteorologas.

\section{Naudojami pradiniai duomenys ir pagrindinės numatomų taikyti metodų grupės}

Pradiniai duomenys -- iš oro prognozių tiekėjų. Tyrimo pradžioje atsisiųsiu
kelias prognozes, kad galėčiau įvertinti formatą ir pradėti vykdyti
nesudėtingas analizes.

\printbibliography

\end{document}
