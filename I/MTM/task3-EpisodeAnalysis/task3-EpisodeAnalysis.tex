\documentclass{article}

\usepackage[L7x,T1]{fontenc}
\usepackage[utf8]{inputenc}
\usepackage[lithuanian]{babel}
\usepackage{hyperref}
\usepackage{cite}

\title{Mokslinių tyrimų metodologija\\ \vspace{4mm} 
Trečioji užduotis -- Mokslo istorijos epizodo analizė}

\author{Motiejus Jakštys}

\date{\today}

\begin{document}
\maketitle

Susipažinti su pasirinktu mokslo istorijos epizodu (konkrečių tyrimų, atradimo
ar teorijos istorija), pageidautina, apimančiu inovatyvius ar prieštaringus
aiškinimus ar interpretacijas. Pakomentuoti, kuo atvejis įdomus.

Rezultatas –trumpas tekstas (tezės). Turi būti paminėta epizodo esmė,
pagrindiniai faktai, kuo epizodas svarbus mokslo raidai arba (ir) kokius
svarbius mokslo metodologijos aspektus jis iliustruoja. Pateikti sistemoje.

\section{Įžanga - Labai Trumpa kompiuterių istorija}

Pirmasis bendrinio panaudojimo ({\it general purpose}) kompiuteris ENIAC buvo
sukurtas 1945 metais, ir kainavo apie \$7M dabartiniais pinigais.

Po 20 metų, septintojo dešimtmečio viduryje, kompiuteriai jau buvo labiau
pigesni, galingesni ir labiau prieinami, nei pirmasis, ir gana papiltę JAV
universitetuose.

\section{Vasaros Projektas}

Kartu su kompiuterių tobulėjimu, tuo metu populiarėjo "dirbtinio intelekto"
tyrimai. Viena įdomesnių problemų -- objektų nuotraukoje klasifikavimas. Vienas
iš MIT dirbtinio intelekto laboratorijos studentų pasiūlė per vasaros
"praktikos" darbą sukurti sistemą, kuri galėtų automatiškai atpažinti žinomus
objektus\cite{papert1966summer}.

\section{Rezultatai}

Projekto rezultatų aprašo rasti nepavyko, tačiau aišku, kad po 50-ies metų ši
sritis yra aktyviai pletojama ir vaizdų klasifikacijos problema iki galo nėra
išspręsta.

Anot \cite{szeliski2010computer}, šis vasaros projektas nepavyko. Tačiau
dešimtmečio viduryje-pabaigoje atsirado daug 3D vaizdo suvokimo tyrimų: kampų,
kraštų, trimačių objektų, kurie vėliau davė pagrindą semantiniam -- objektų --
suvokimui.

\bibliography{bib}{}
\bibliographystyle{plain}

\end{document}
