\documentclass{article}

\usepackage[L7x,T1]{fontenc}
\usepackage[utf8]{inputenc}
\usepackage[lithuanian]{babel}
\usepackage{hyperref}

\title{Mokslinių tyrimų metodologija\\ \vspace{4mm} 
Antroji užduotis -- magistro darbų problemos}

\author{Motiejus Jakštys}

\date{\today}

\begin{document}
\maketitle

\section{TYRIMO PROBLEMA}

Sujungti kelias paviršiaus nuotraukas yra sudėtinga, kai paviršius yra
monotoniškas. Įprastinės priemonės reikalauja padėti aiškiai matomus daiktus
vietoj orientyrų, tačiau logistiškai tai yra sudėtinga. Šiame darbe
įvertinsime, ar kaip orientyrus naudojant nedidelius, mažos kainos bepiločius
orlaivius šį procesą galima paefektyvinti.

\section{TYRIMO TIKSLAS}

Siekiame padidinti tikslumą fotografuojant monotoniškus paviršius kiek įmanoma
mažai padidinant kainą. Tyrimo tikslas -- ištirti ir aprašyti kainos santykį
lyginant su egzistuojančiams metodais.

\section{TYRIMO KLAUSIMAS}

Ar naudojant mažus bepiločius orlaivius kaip orientyrus galima pagerinti
monotoniško paviršiaus ortofoto kokybę?

\section{TYRIMO TEMA}

Kiek pagerinamas monotoniško paviršiaus fotografavimo tikslumas naudojant mažus
bepiločius orlaivius kaip orientyrus?

\end{document}
